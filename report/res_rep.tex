\subsection{Three-player Intersection}

In this section, we are going to reproduce paper's \cite{fridovichkeil2020} results using a similar scenario to the three player intersection.

\subsubsection{Problem Description}

The scenario consists of 2 cars and a pedestrian that must cross and intersection avoiding collisions. The problem's time horizon is $T=5s$ and the time discretization is $\Delta t = 0.1s$.

\subsubsection{System Dynamics}

The three cars' dynamics are modeled after kinematic bicycle model:

$$
\dot{p}_{x,i} = v_i \cos(\theta_i), \quad \dot{p}_{y,i} = v_i \sin(\theta_i), \quad \dot{\theta}_i = \frac{v_i \tan(\phi_i)}{L_i}, \quad \dot{\phi}_i = \psi_i, \quad \dot{v}_i = a_i 	
$$

\noindent where $p_{x,i}, \; p_{y,i}$ are each car's position coordinates, $\theta_i$ is each car's heading angle, $u_i$ the linear velocity and $\phi_i$ the front wheel's steering angle. $L_i$ is the inter-axle distance, and input $u_i = (\psi_i, \alpha_i)$ is the front wheel angular rate and longitudinal acceleration, respectively \\ 

\noindent The pedestrian is modeled after the kinematic unicycle model:

$$
\dot{p}_x &= v \cos(\theta), \ \dot{p}_y &= v \sin(\theta), \ \dot{\theta} &= \omega, \ \dot{v} &= a 
$$

\noindent where $p_x, \; p_y$ are the pedestrian's position coordinates, $\theta$ their heading angle and $u$ their linear velocity. Pedestrian's control input is $u_{ped} = [\omega, \alpha]^T$ (angular velocity and longitudinal acceleration). \\

\noindent Thus, our system is described by a 14-D state space model. There are 5 state variables for each car and 4 for the pedestrian.

\subsubsection{Cost Functions}

Each player's cost is a weighted sum of penalties for:

\begin{itemize}
	\item \textit{Lane Center and Boundary:} Penalty for deviation from the center and exit of the lane
	
	\item \textit{Nominal Speed and Speed Bounds:} Enforcement of velocity $u_{ref, i}$ and respect of speed limits
	
	\item \textit{Proximity Cost:} Avoid close proximity of the players, because that can result to a collision.
\end{itemize}

\subsubsection{Our Scenario and Results}

In our simulation we defined a reduced system with two players. Those agents' dynamics adhere to bicycle kinematic model (they are cars). Thus, our system is 10-D. The two lanes are straight lines for implementation simplicity. The objective is crossing the intersection without collisions. \\

\noindent Our simulation was implemented utilizing \cite{iLQGames}, a software library in Julia developed to solve games described by paper \cite{fridovichkeil2020}. \\
\noindent There, one can defined the system's dynamics as presented in the code snippet from file \textit{intersection.jl} below:

\begin{minted}{julia}
using iLQGames
import iLQGames: dx, xyindex

nx, nu, dt, game_horizon = 10, 4, 0.1, 500
L1, L2 = 2.5, 2.9 # wheelbases

struct Intersection <: ControlSystem{dt, nx, nu} end

dx(cs::Intersection, x, u, t) = SVector(
    # Car1 (Compact) dynamics - Bicycle kinematic model
    # [px, py, θ, v, φ]
    x[4]cos(x[3]),
    x[4]sin(x[3]),
    x[4]*tan(x[5])/L1,
    u[2],
    u[1],
    # Car2 (SUV) dynamics - Bicycle kinematic model
    # [px, py, θ, v, φ]
    x[9]cos(x[8]),
    x[9]sin(x[8]),
    x[9]*tan(x[10])/L2,
    u[4],
    u[3]
)

# Specify players' positions
xyindex(::Intersection) = ((1,2), (6,7))
\end{minted}

\noindent The cost functions that we implemented the ones presented in \cite{fridovichkeil2020}. Our simulation parameters are:

\begin{itemize}
	\item $d_{prox} = 0.5$
	
	\item $t_{goal} = 2.0$
	
	\item $d_{lane} = 1.5$
\end{itemize}

\noindent Our cars have different size specified through their wheelbase, 2.5 and 2.9 meters respectively. \\

\noindent Here is their resulting trajectory: \\

\begin{figure}[h!]
	\centering
	\includegraphics[width=0.8\textwidth]{../simulation/results/intersection.png}
	\caption{Two player intersection}
\end{figure}

\noindent The corresponding gif file illustrates that there is no collision between our agents. Our algorithm indeed converges to a Nash equilibrium solving this simple game.